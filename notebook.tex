
% Default to the notebook output style

    


% Inherit from the specified cell style.




    
\documentclass[11pt]{article}

    
    
    \usepackage[T1]{fontenc}
    % Nicer default font (+ math font) than Computer Modern for most use cases
    \usepackage{mathpazo}

    % Basic figure setup, for now with no caption control since it's done
    % automatically by Pandoc (which extracts ![](path) syntax from Markdown).
    \usepackage{graphicx}
    % We will generate all images so they have a width \maxwidth. This means
    % that they will get their normal width if they fit onto the page, but
    % are scaled down if they would overflow the margins.
    \makeatletter
    \def\maxwidth{\ifdim\Gin@nat@width>\linewidth\linewidth
    \else\Gin@nat@width\fi}
    \makeatother
    \let\Oldincludegraphics\includegraphics
    % Set max figure width to be 80% of text width, for now hardcoded.
    \renewcommand{\includegraphics}[1]{\Oldincludegraphics[width=.8\maxwidth]{#1}}
    % Ensure that by default, figures have no caption (until we provide a
    % proper Figure object with a Caption API and a way to capture that
    % in the conversion process - todo).
    \usepackage{caption}
    \DeclareCaptionLabelFormat{nolabel}{}
    \captionsetup{labelformat=nolabel}

    \usepackage{adjustbox} % Used to constrain images to a maximum size 
    \usepackage{xcolor} % Allow colors to be defined
    \usepackage{enumerate} % Needed for markdown enumerations to work
    \usepackage{geometry} % Used to adjust the document margins
    \usepackage{amsmath} % Equations
    \usepackage{amssymb} % Equations
    \usepackage{textcomp} % defines textquotesingle
    % Hack from http://tex.stackexchange.com/a/47451/13684:
    \AtBeginDocument{%
        \def\PYZsq{\textquotesingle}% Upright quotes in Pygmentized code
    }
    \usepackage{upquote} % Upright quotes for verbatim code
    \usepackage{eurosym} % defines \euro
    \usepackage[mathletters]{ucs} % Extended unicode (utf-8) support
    \usepackage[utf8x]{inputenc} % Allow utf-8 characters in the tex document
    \usepackage{fancyvrb} % verbatim replacement that allows latex
    \usepackage{grffile} % extends the file name processing of package graphics 
                         % to support a larger range 
    % The hyperref package gives us a pdf with properly built
    % internal navigation ('pdf bookmarks' for the table of contents,
    % internal cross-reference links, web links for URLs, etc.)
    \usepackage{hyperref}
    \usepackage{longtable} % longtable support required by pandoc >1.10
    \usepackage{booktabs}  % table support for pandoc > 1.12.2
    \usepackage[inline]{enumitem} % IRkernel/repr support (it uses the enumerate* environment)
    \usepackage[normalem]{ulem} % ulem is needed to support strikethroughs (\sout)
                                % normalem makes italics be italics, not underlines
    

    
    
    % Colors for the hyperref package
    \definecolor{urlcolor}{rgb}{0,.145,.698}
    \definecolor{linkcolor}{rgb}{.71,0.21,0.01}
    \definecolor{citecolor}{rgb}{.12,.54,.11}

    % ANSI colors
    \definecolor{ansi-black}{HTML}{3E424D}
    \definecolor{ansi-black-intense}{HTML}{282C36}
    \definecolor{ansi-red}{HTML}{E75C58}
    \definecolor{ansi-red-intense}{HTML}{B22B31}
    \definecolor{ansi-green}{HTML}{00A250}
    \definecolor{ansi-green-intense}{HTML}{007427}
    \definecolor{ansi-yellow}{HTML}{DDB62B}
    \definecolor{ansi-yellow-intense}{HTML}{B27D12}
    \definecolor{ansi-blue}{HTML}{208FFB}
    \definecolor{ansi-blue-intense}{HTML}{0065CA}
    \definecolor{ansi-magenta}{HTML}{D160C4}
    \definecolor{ansi-magenta-intense}{HTML}{A03196}
    \definecolor{ansi-cyan}{HTML}{60C6C8}
    \definecolor{ansi-cyan-intense}{HTML}{258F8F}
    \definecolor{ansi-white}{HTML}{C5C1B4}
    \definecolor{ansi-white-intense}{HTML}{A1A6B2}

    % commands and environments needed by pandoc snippets
    % extracted from the output of `pandoc -s`
    \providecommand{\tightlist}{%
      \setlength{\itemsep}{0pt}\setlength{\parskip}{0pt}}
    \DefineVerbatimEnvironment{Highlighting}{Verbatim}{commandchars=\\\{\}}
    % Add ',fontsize=\small' for more characters per line
    \newenvironment{Shaded}{}{}
    \newcommand{\KeywordTok}[1]{\textcolor[rgb]{0.00,0.44,0.13}{\textbf{{#1}}}}
    \newcommand{\DataTypeTok}[1]{\textcolor[rgb]{0.56,0.13,0.00}{{#1}}}
    \newcommand{\DecValTok}[1]{\textcolor[rgb]{0.25,0.63,0.44}{{#1}}}
    \newcommand{\BaseNTok}[1]{\textcolor[rgb]{0.25,0.63,0.44}{{#1}}}
    \newcommand{\FloatTok}[1]{\textcolor[rgb]{0.25,0.63,0.44}{{#1}}}
    \newcommand{\CharTok}[1]{\textcolor[rgb]{0.25,0.44,0.63}{{#1}}}
    \newcommand{\StringTok}[1]{\textcolor[rgb]{0.25,0.44,0.63}{{#1}}}
    \newcommand{\CommentTok}[1]{\textcolor[rgb]{0.38,0.63,0.69}{\textit{{#1}}}}
    \newcommand{\OtherTok}[1]{\textcolor[rgb]{0.00,0.44,0.13}{{#1}}}
    \newcommand{\AlertTok}[1]{\textcolor[rgb]{1.00,0.00,0.00}{\textbf{{#1}}}}
    \newcommand{\FunctionTok}[1]{\textcolor[rgb]{0.02,0.16,0.49}{{#1}}}
    \newcommand{\RegionMarkerTok}[1]{{#1}}
    \newcommand{\ErrorTok}[1]{\textcolor[rgb]{1.00,0.00,0.00}{\textbf{{#1}}}}
    \newcommand{\NormalTok}[1]{{#1}}
    
    % Additional commands for more recent versions of Pandoc
    \newcommand{\ConstantTok}[1]{\textcolor[rgb]{0.53,0.00,0.00}{{#1}}}
    \newcommand{\SpecialCharTok}[1]{\textcolor[rgb]{0.25,0.44,0.63}{{#1}}}
    \newcommand{\VerbatimStringTok}[1]{\textcolor[rgb]{0.25,0.44,0.63}{{#1}}}
    \newcommand{\SpecialStringTok}[1]{\textcolor[rgb]{0.73,0.40,0.53}{{#1}}}
    \newcommand{\ImportTok}[1]{{#1}}
    \newcommand{\DocumentationTok}[1]{\textcolor[rgb]{0.73,0.13,0.13}{\textit{{#1}}}}
    \newcommand{\AnnotationTok}[1]{\textcolor[rgb]{0.38,0.63,0.69}{\textbf{\textit{{#1}}}}}
    \newcommand{\CommentVarTok}[1]{\textcolor[rgb]{0.38,0.63,0.69}{\textbf{\textit{{#1}}}}}
    \newcommand{\VariableTok}[1]{\textcolor[rgb]{0.10,0.09,0.49}{{#1}}}
    \newcommand{\ControlFlowTok}[1]{\textcolor[rgb]{0.00,0.44,0.13}{\textbf{{#1}}}}
    \newcommand{\OperatorTok}[1]{\textcolor[rgb]{0.40,0.40,0.40}{{#1}}}
    \newcommand{\BuiltInTok}[1]{{#1}}
    \newcommand{\ExtensionTok}[1]{{#1}}
    \newcommand{\PreprocessorTok}[1]{\textcolor[rgb]{0.74,0.48,0.00}{{#1}}}
    \newcommand{\AttributeTok}[1]{\textcolor[rgb]{0.49,0.56,0.16}{{#1}}}
    \newcommand{\InformationTok}[1]{\textcolor[rgb]{0.38,0.63,0.69}{\textbf{\textit{{#1}}}}}
    \newcommand{\WarningTok}[1]{\textcolor[rgb]{0.38,0.63,0.69}{\textbf{\textit{{#1}}}}}
    
    
    % Define a nice break command that doesn't care if a line doesn't already
    % exist.
    \def\br{\hspace*{\fill} \\* }
    % Math Jax compatability definitions
    \def\gt{>}
    \def\lt{<}
    % Document parameters
    \title{template\_AST3001-assignment1-SOLVED}
    
    
    

    % Pygments definitions
    
\makeatletter
\def\PY@reset{\let\PY@it=\relax \let\PY@bf=\relax%
    \let\PY@ul=\relax \let\PY@tc=\relax%
    \let\PY@bc=\relax \let\PY@ff=\relax}
\def\PY@tok#1{\csname PY@tok@#1\endcsname}
\def\PY@toks#1+{\ifx\relax#1\empty\else%
    \PY@tok{#1}\expandafter\PY@toks\fi}
\def\PY@do#1{\PY@bc{\PY@tc{\PY@ul{%
    \PY@it{\PY@bf{\PY@ff{#1}}}}}}}
\def\PY#1#2{\PY@reset\PY@toks#1+\relax+\PY@do{#2}}

\expandafter\def\csname PY@tok@w\endcsname{\def\PY@tc##1{\textcolor[rgb]{0.73,0.73,0.73}{##1}}}
\expandafter\def\csname PY@tok@c\endcsname{\let\PY@it=\textit\def\PY@tc##1{\textcolor[rgb]{0.25,0.50,0.50}{##1}}}
\expandafter\def\csname PY@tok@cp\endcsname{\def\PY@tc##1{\textcolor[rgb]{0.74,0.48,0.00}{##1}}}
\expandafter\def\csname PY@tok@k\endcsname{\let\PY@bf=\textbf\def\PY@tc##1{\textcolor[rgb]{0.00,0.50,0.00}{##1}}}
\expandafter\def\csname PY@tok@kp\endcsname{\def\PY@tc##1{\textcolor[rgb]{0.00,0.50,0.00}{##1}}}
\expandafter\def\csname PY@tok@kt\endcsname{\def\PY@tc##1{\textcolor[rgb]{0.69,0.00,0.25}{##1}}}
\expandafter\def\csname PY@tok@o\endcsname{\def\PY@tc##1{\textcolor[rgb]{0.40,0.40,0.40}{##1}}}
\expandafter\def\csname PY@tok@ow\endcsname{\let\PY@bf=\textbf\def\PY@tc##1{\textcolor[rgb]{0.67,0.13,1.00}{##1}}}
\expandafter\def\csname PY@tok@nb\endcsname{\def\PY@tc##1{\textcolor[rgb]{0.00,0.50,0.00}{##1}}}
\expandafter\def\csname PY@tok@nf\endcsname{\def\PY@tc##1{\textcolor[rgb]{0.00,0.00,1.00}{##1}}}
\expandafter\def\csname PY@tok@nc\endcsname{\let\PY@bf=\textbf\def\PY@tc##1{\textcolor[rgb]{0.00,0.00,1.00}{##1}}}
\expandafter\def\csname PY@tok@nn\endcsname{\let\PY@bf=\textbf\def\PY@tc##1{\textcolor[rgb]{0.00,0.00,1.00}{##1}}}
\expandafter\def\csname PY@tok@ne\endcsname{\let\PY@bf=\textbf\def\PY@tc##1{\textcolor[rgb]{0.82,0.25,0.23}{##1}}}
\expandafter\def\csname PY@tok@nv\endcsname{\def\PY@tc##1{\textcolor[rgb]{0.10,0.09,0.49}{##1}}}
\expandafter\def\csname PY@tok@no\endcsname{\def\PY@tc##1{\textcolor[rgb]{0.53,0.00,0.00}{##1}}}
\expandafter\def\csname PY@tok@nl\endcsname{\def\PY@tc##1{\textcolor[rgb]{0.63,0.63,0.00}{##1}}}
\expandafter\def\csname PY@tok@ni\endcsname{\let\PY@bf=\textbf\def\PY@tc##1{\textcolor[rgb]{0.60,0.60,0.60}{##1}}}
\expandafter\def\csname PY@tok@na\endcsname{\def\PY@tc##1{\textcolor[rgb]{0.49,0.56,0.16}{##1}}}
\expandafter\def\csname PY@tok@nt\endcsname{\let\PY@bf=\textbf\def\PY@tc##1{\textcolor[rgb]{0.00,0.50,0.00}{##1}}}
\expandafter\def\csname PY@tok@nd\endcsname{\def\PY@tc##1{\textcolor[rgb]{0.67,0.13,1.00}{##1}}}
\expandafter\def\csname PY@tok@s\endcsname{\def\PY@tc##1{\textcolor[rgb]{0.73,0.13,0.13}{##1}}}
\expandafter\def\csname PY@tok@sd\endcsname{\let\PY@it=\textit\def\PY@tc##1{\textcolor[rgb]{0.73,0.13,0.13}{##1}}}
\expandafter\def\csname PY@tok@si\endcsname{\let\PY@bf=\textbf\def\PY@tc##1{\textcolor[rgb]{0.73,0.40,0.53}{##1}}}
\expandafter\def\csname PY@tok@se\endcsname{\let\PY@bf=\textbf\def\PY@tc##1{\textcolor[rgb]{0.73,0.40,0.13}{##1}}}
\expandafter\def\csname PY@tok@sr\endcsname{\def\PY@tc##1{\textcolor[rgb]{0.73,0.40,0.53}{##1}}}
\expandafter\def\csname PY@tok@ss\endcsname{\def\PY@tc##1{\textcolor[rgb]{0.10,0.09,0.49}{##1}}}
\expandafter\def\csname PY@tok@sx\endcsname{\def\PY@tc##1{\textcolor[rgb]{0.00,0.50,0.00}{##1}}}
\expandafter\def\csname PY@tok@m\endcsname{\def\PY@tc##1{\textcolor[rgb]{0.40,0.40,0.40}{##1}}}
\expandafter\def\csname PY@tok@gh\endcsname{\let\PY@bf=\textbf\def\PY@tc##1{\textcolor[rgb]{0.00,0.00,0.50}{##1}}}
\expandafter\def\csname PY@tok@gu\endcsname{\let\PY@bf=\textbf\def\PY@tc##1{\textcolor[rgb]{0.50,0.00,0.50}{##1}}}
\expandafter\def\csname PY@tok@gd\endcsname{\def\PY@tc##1{\textcolor[rgb]{0.63,0.00,0.00}{##1}}}
\expandafter\def\csname PY@tok@gi\endcsname{\def\PY@tc##1{\textcolor[rgb]{0.00,0.63,0.00}{##1}}}
\expandafter\def\csname PY@tok@gr\endcsname{\def\PY@tc##1{\textcolor[rgb]{1.00,0.00,0.00}{##1}}}
\expandafter\def\csname PY@tok@ge\endcsname{\let\PY@it=\textit}
\expandafter\def\csname PY@tok@gs\endcsname{\let\PY@bf=\textbf}
\expandafter\def\csname PY@tok@gp\endcsname{\let\PY@bf=\textbf\def\PY@tc##1{\textcolor[rgb]{0.00,0.00,0.50}{##1}}}
\expandafter\def\csname PY@tok@go\endcsname{\def\PY@tc##1{\textcolor[rgb]{0.53,0.53,0.53}{##1}}}
\expandafter\def\csname PY@tok@gt\endcsname{\def\PY@tc##1{\textcolor[rgb]{0.00,0.27,0.87}{##1}}}
\expandafter\def\csname PY@tok@err\endcsname{\def\PY@bc##1{\setlength{\fboxsep}{0pt}\fcolorbox[rgb]{1.00,0.00,0.00}{1,1,1}{\strut ##1}}}
\expandafter\def\csname PY@tok@kc\endcsname{\let\PY@bf=\textbf\def\PY@tc##1{\textcolor[rgb]{0.00,0.50,0.00}{##1}}}
\expandafter\def\csname PY@tok@kd\endcsname{\let\PY@bf=\textbf\def\PY@tc##1{\textcolor[rgb]{0.00,0.50,0.00}{##1}}}
\expandafter\def\csname PY@tok@kn\endcsname{\let\PY@bf=\textbf\def\PY@tc##1{\textcolor[rgb]{0.00,0.50,0.00}{##1}}}
\expandafter\def\csname PY@tok@kr\endcsname{\let\PY@bf=\textbf\def\PY@tc##1{\textcolor[rgb]{0.00,0.50,0.00}{##1}}}
\expandafter\def\csname PY@tok@bp\endcsname{\def\PY@tc##1{\textcolor[rgb]{0.00,0.50,0.00}{##1}}}
\expandafter\def\csname PY@tok@fm\endcsname{\def\PY@tc##1{\textcolor[rgb]{0.00,0.00,1.00}{##1}}}
\expandafter\def\csname PY@tok@vc\endcsname{\def\PY@tc##1{\textcolor[rgb]{0.10,0.09,0.49}{##1}}}
\expandafter\def\csname PY@tok@vg\endcsname{\def\PY@tc##1{\textcolor[rgb]{0.10,0.09,0.49}{##1}}}
\expandafter\def\csname PY@tok@vi\endcsname{\def\PY@tc##1{\textcolor[rgb]{0.10,0.09,0.49}{##1}}}
\expandafter\def\csname PY@tok@vm\endcsname{\def\PY@tc##1{\textcolor[rgb]{0.10,0.09,0.49}{##1}}}
\expandafter\def\csname PY@tok@sa\endcsname{\def\PY@tc##1{\textcolor[rgb]{0.73,0.13,0.13}{##1}}}
\expandafter\def\csname PY@tok@sb\endcsname{\def\PY@tc##1{\textcolor[rgb]{0.73,0.13,0.13}{##1}}}
\expandafter\def\csname PY@tok@sc\endcsname{\def\PY@tc##1{\textcolor[rgb]{0.73,0.13,0.13}{##1}}}
\expandafter\def\csname PY@tok@dl\endcsname{\def\PY@tc##1{\textcolor[rgb]{0.73,0.13,0.13}{##1}}}
\expandafter\def\csname PY@tok@s2\endcsname{\def\PY@tc##1{\textcolor[rgb]{0.73,0.13,0.13}{##1}}}
\expandafter\def\csname PY@tok@sh\endcsname{\def\PY@tc##1{\textcolor[rgb]{0.73,0.13,0.13}{##1}}}
\expandafter\def\csname PY@tok@s1\endcsname{\def\PY@tc##1{\textcolor[rgb]{0.73,0.13,0.13}{##1}}}
\expandafter\def\csname PY@tok@mb\endcsname{\def\PY@tc##1{\textcolor[rgb]{0.40,0.40,0.40}{##1}}}
\expandafter\def\csname PY@tok@mf\endcsname{\def\PY@tc##1{\textcolor[rgb]{0.40,0.40,0.40}{##1}}}
\expandafter\def\csname PY@tok@mh\endcsname{\def\PY@tc##1{\textcolor[rgb]{0.40,0.40,0.40}{##1}}}
\expandafter\def\csname PY@tok@mi\endcsname{\def\PY@tc##1{\textcolor[rgb]{0.40,0.40,0.40}{##1}}}
\expandafter\def\csname PY@tok@il\endcsname{\def\PY@tc##1{\textcolor[rgb]{0.40,0.40,0.40}{##1}}}
\expandafter\def\csname PY@tok@mo\endcsname{\def\PY@tc##1{\textcolor[rgb]{0.40,0.40,0.40}{##1}}}
\expandafter\def\csname PY@tok@ch\endcsname{\let\PY@it=\textit\def\PY@tc##1{\textcolor[rgb]{0.25,0.50,0.50}{##1}}}
\expandafter\def\csname PY@tok@cm\endcsname{\let\PY@it=\textit\def\PY@tc##1{\textcolor[rgb]{0.25,0.50,0.50}{##1}}}
\expandafter\def\csname PY@tok@cpf\endcsname{\let\PY@it=\textit\def\PY@tc##1{\textcolor[rgb]{0.25,0.50,0.50}{##1}}}
\expandafter\def\csname PY@tok@c1\endcsname{\let\PY@it=\textit\def\PY@tc##1{\textcolor[rgb]{0.25,0.50,0.50}{##1}}}
\expandafter\def\csname PY@tok@cs\endcsname{\let\PY@it=\textit\def\PY@tc##1{\textcolor[rgb]{0.25,0.50,0.50}{##1}}}

\def\PYZbs{\char`\\}
\def\PYZus{\char`\_}
\def\PYZob{\char`\{}
\def\PYZcb{\char`\}}
\def\PYZca{\char`\^}
\def\PYZam{\char`\&}
\def\PYZlt{\char`\<}
\def\PYZgt{\char`\>}
\def\PYZsh{\char`\#}
\def\PYZpc{\char`\%}
\def\PYZdl{\char`\$}
\def\PYZhy{\char`\-}
\def\PYZsq{\char`\'}
\def\PYZdq{\char`\"}
\def\PYZti{\char`\~}
% for compatibility with earlier versions
\def\PYZat{@}
\def\PYZlb{[}
\def\PYZrb{]}
\makeatother


    % Exact colors from NB
    \definecolor{incolor}{rgb}{0.0, 0.0, 0.5}
    \definecolor{outcolor}{rgb}{0.545, 0.0, 0.0}



    
    % Prevent overflowing lines due to hard-to-break entities
    \sloppy 
    % Setup hyperref package
    \hypersetup{
      breaklinks=true,  % so long urls are correctly broken across lines
      colorlinks=true,
      urlcolor=urlcolor,
      linkcolor=linkcolor,
      citecolor=citecolor,
      }
    % Slightly bigger margins than the latex defaults
    
    \geometry{verbose,tmargin=1in,bmargin=1in,lmargin=1in,rmargin=1in}
    
    

    \begin{document}
    
    
    \maketitle
    
    

    
    \section{Welcome to Assignment 1 for
AST3001.}\label{welcome-to-assignment-1-for-ast3001.}

For those unfamiliar with Jupyter Notebook, each window can be run by
simply pressing 'shift'+'enter' keys. To add another cell to run more
code, simply click the insert button at the top of the browser, and
'Insert Cell Above/Below'. You can find some tutorials at this website:
https://github.com/jupyter/jupyter/wiki/A-gallery-of-interesting-Jupyter-Notebooks.

Any issues or quetions please email swebb@swin.edu.au with the subject
title containing 'AST3001 assignment - '

ENJOY!

    First let's start by loading all our modules for this session, because
we are running on an online interactive session we'll need to install
the modules first (note: when running python on your local computer you
won't need to do this).

    \begin{Verbatim}[commandchars=\\\{\}]
{\color{incolor}In [{\color{incolor}1}]:} \PY{o}{!}pip install astropy 
        \PY{o}{!}pip install matplotlib 
        \PY{o}{!}pip install numpy
\end{Verbatim}


    \begin{Verbatim}[commandchars=\\\{\}]
Requirement already satisfied: astropy in /anaconda3/lib/python3.6/site-packages (3.0.5)
Requirement already satisfied: numpy>=1.10.0 in /anaconda3/lib/python3.6/site-packages (from astropy) (1.16.1)
\textcolor{ansi-yellow}{WARNING: You are using pip version 19.2.1, however version 19.2.2 is available.
You should consider upgrading via the 'pip install --upgrade pip' command.}
Requirement already satisfied: matplotlib in /anaconda3/lib/python3.6/site-packages (3.1.1)
Requirement already satisfied: pyparsing!=2.0.4,!=2.1.2,!=2.1.6,>=2.0.1 in /anaconda3/lib/python3.6/site-packages (from matplotlib) (2.2.0)
Requirement already satisfied: cycler>=0.10 in /anaconda3/lib/python3.6/site-packages (from matplotlib) (0.10.0)
Requirement already satisfied: kiwisolver>=1.0.1 in /anaconda3/lib/python3.6/site-packages (from matplotlib) (1.0.1)
Requirement already satisfied: numpy>=1.11 in /anaconda3/lib/python3.6/site-packages (from matplotlib) (1.16.1)
Requirement already satisfied: python-dateutil>=2.1 in /anaconda3/lib/python3.6/site-packages (from matplotlib) (2.6.1)
Requirement already satisfied: six in /anaconda3/lib/python3.6/site-packages (from cycler>=0.10->matplotlib) (1.11.0)
Requirement already satisfied: setuptools in /anaconda3/lib/python3.6/site-packages (from kiwisolver>=1.0.1->matplotlib) (40.6.2)
\textcolor{ansi-yellow}{WARNING: You are using pip version 19.2.1, however version 19.2.2 is available.
You should consider upgrading via the 'pip install --upgrade pip' command.}
Requirement already satisfied: numpy in /anaconda3/lib/python3.6/site-packages (1.16.1)
\textcolor{ansi-yellow}{WARNING: You are using pip version 19.2.1, however version 19.2.2 is available.
You should consider upgrading via the 'pip install --upgrade pip' command.}

    \end{Verbatim}

    \begin{Verbatim}[commandchars=\\\{\}]
{\color{incolor}In [{\color{incolor}2}]:} \PY{k+kn}{from} \PY{n+nn}{astropy}\PY{n+nn}{.}\PY{n+nn}{io} \PY{k}{import} \PY{n}{fits}
        \PY{k+kn}{import} \PY{n+nn}{matplotlib}\PY{n+nn}{.}\PY{n+nn}{pyplot} \PY{k}{as} \PY{n+nn}{plt}
        \PY{k+kn}{import} \PY{n+nn}{numpy} \PY{k}{as} \PY{n+nn}{np}
        \PY{k+kn}{from} \PY{n+nn}{spec\PYZus{}tools} \PY{k}{import} \PY{o}{*}
        \PY{k+kn}{import} \PY{n+nn}{subprocess}
        \PY{o}{\PYZpc{}}\PY{k}{matplotlib} inline
\end{Verbatim}


    Great. For this assignment we will be looking at a star HD122563 and
comparing it to a Solar twin star HD147513.

The aim of this assignment is to get an idea of concepts such as
spectral type and features, radial velocity and metallicity.

    \section{Prep: First we need to read in our
data}\label{prep-first-we-need-to-read-in-our-data}

We can do this using a pre-defined function called 'open\_file'. This
function takes the id of the star as an argument, and assigns the
wavelength and flux arrays to the output of this function. Simply 'run'
the code box below by selecting the box and pressing 'shift'+'enter'

    \begin{Verbatim}[commandchars=\\\{\}]
{\color{incolor}In [{\color{incolor}14}]:} \PY{n}{star\PYZus{}id} \PY{o}{=} \PY{l+s+s1}{\PYZsq{}}\PY{l+s+s1}{HD122563}\PY{l+s+s1}{\PYZsq{}}
         \PY{n}{wave}\PY{p}{,}\PY{n}{flux} \PY{o}{=} \PY{n}{open\PYZus{}file}\PY{p}{(}\PY{n}{star\PYZus{}id}\PY{p}{)}
         
         \PY{n}{solar\PYZus{}twin} \PY{o}{=} \PY{l+s+s1}{\PYZsq{}}\PY{l+s+s1}{HD147513}\PY{l+s+s1}{\PYZsq{}}
         \PY{n}{sol\PYZus{}wave}\PY{p}{,}\PY{n}{sol\PYZus{}flux} \PY{o}{=} \PY{n}{open\PYZus{}file}\PY{p}{(}\PY{n}{solar\PYZus{}twin}\PY{p}{)}
\end{Verbatim}


    Now we've read in the data, we can plot the flux and wavelength of these
spectra using the matplotlib functions below. Again simply select the
box and press 'shift'+'enter'

    Use the zoom to rectangle icon Above the plot to zoom in on some
spectral features. Note - The home button resets to the original view
Alternatively, you can remove the '\#' from the bottom lines to zoom
into sections the following command to automatically zoom in on a
wavelength range.

    \begin{Verbatim}[commandchars=\\\{\}]
{\color{incolor}In [{\color{incolor}15}]:} \PY{o}{\PYZpc{}}\PY{k}{matplotlib} notebook
         \PY{c+c1}{\PYZsh{}Plots a figure and a set of axes for which we can use to plot our data.}
         \PY{n}{fig}\PY{p}{,}\PY{n}{ax} \PY{o}{=} \PY{n}{plt}\PY{o}{.}\PY{n}{subplots}\PY{p}{(}\PY{l+m+mi}{1}\PY{p}{,}\PY{l+m+mi}{2}\PY{p}{,}\PY{n}{figsize}\PY{o}{=}\PY{p}{(}\PY{l+m+mi}{10}\PY{p}{,}\PY{l+m+mi}{5}\PY{p}{)}\PY{p}{)} \PY{c+c1}{\PYZsh{} this is making a figure for us to put two plots in}
         
         \PY{c+c1}{\PYZsh{}Plot spectrum for HD122563}
         \PY{n}{ax}\PY{p}{[}\PY{l+m+mi}{0}\PY{p}{]}\PY{o}{.}\PY{n}{plot}\PY{p}{(}\PY{n}{wave}\PY{p}{,}\PY{n}{flux}\PY{p}{)} 
         \PY{n}{ax}\PY{p}{[}\PY{l+m+mi}{0}\PY{p}{]}\PY{o}{.}\PY{n}{set\PYZus{}ylim}\PY{p}{(}\PY{l+m+mi}{0}\PY{p}{,}\PY{l+m+mf}{1.1}\PY{p}{)}
         \PY{n}{ax}\PY{p}{[}\PY{l+m+mi}{0}\PY{p}{]}\PY{o}{.}\PY{n}{set\PYZus{}title}\PY{p}{(}\PY{l+s+s1}{\PYZsq{}}\PY{l+s+s1}{HD122563}\PY{l+s+s1}{\PYZsq{}}\PY{p}{)}
         
         \PY{c+c1}{\PYZsh{}Do the same for HD147513}
         \PY{n}{ax}\PY{p}{[}\PY{l+m+mi}{1}\PY{p}{]}\PY{o}{.}\PY{n}{plot}\PY{p}{(}\PY{n}{sol\PYZus{}wave}\PY{p}{,}\PY{n}{sol\PYZus{}flux}\PY{p}{)}
         \PY{n}{ax}\PY{p}{[}\PY{l+m+mi}{1}\PY{p}{]}\PY{o}{.}\PY{n}{set\PYZus{}ylim}\PY{p}{(}\PY{l+m+mi}{0}\PY{p}{,}\PY{l+m+mf}{1.1}\PY{p}{)}
         \PY{n}{ax}\PY{p}{[}\PY{l+m+mi}{1}\PY{p}{]}\PY{o}{.}\PY{n}{set\PYZus{}title}\PY{p}{(}\PY{l+s+s1}{\PYZsq{}}\PY{l+s+s1}{HD147513}\PY{l+s+s1}{\PYZsq{}}\PY{p}{)}
         
         \PY{c+c1}{\PYZsh{}\PYZsh{}\PYZsh{}\PYZsh{}\PYZsh{} FOR  HDHD122563 \PYZsh{}\PYZsh{}\PYZsh{}\PYZsh{}\PYZsh{}}
         
         \PY{c+c1}{\PYZsh{} ax[0].set\PYZus{}xlim(3950,4000)}
         
         \PY{c+c1}{\PYZsh{}\PYZsh{}\PYZsh{}\PYZsh{}\PYZsh{} FOR HD147513 \PYZsh{}\PYZsh{}\PYZsh{}\PYZsh{}\PYZsh{}}
         
         \PY{c+c1}{\PYZsh{}ax[1].set\PYZus{}xlim(3950,4000)}
\end{Verbatim}


    
    \begin{verbatim}
<IPython.core.display.Javascript object>
    \end{verbatim}

    
    
    \begin{verbatim}
<IPython.core.display.HTML object>
    \end{verbatim}

    
\begin{Verbatim}[commandchars=\\\{\}]
{\color{outcolor}Out[{\color{outcolor}15}]:} Text(0.5, 1.0, 'HD147513')
\end{Verbatim}
            
    \section{PART 1 - FINDING SPECTRAL
TYPE}\label{part-1---finding-spectral-type}

Using the plots above, let's look at how do these two spectra differ
qualitatively.

Question 1: What is the spectral type of HD147513? Comparitively, what
is the spectral type of HD122563? Hint: What significant spectral
features come and go with OBAFGKM spectral types? Make reference to
these spectral features to justify your answer.

     The solar twin is a G type star, HD122563 is a metal poor red giant
star approx F8 

    Question 2: Name two things that contribute to the width/profile of an
absorption line?

     1. the number of atoms, more atoms the more photons that are absorbed
and the boarder/wider the absorbtion line. 2. Lines can be
widened/boarden from doppler shift. 

    \section{Part 2 - RADIAL VELOCITY}\label{part-2---radial-velocity}

    Let's use this plot to measure the radial velocity of the star. You can
zoom in on a line feature (by setting the appropriate x-range limits)
and then compare the observed wavelength for the line feature with the
rest wavelength of that feature to find the radial velocity of the star.
Hold the cursor over the plot to see the wavelength of the absorption
feature. We can use the rest frame Balmer series lines to compare to
calculate the radial velocity of this star.

    \begin{Verbatim}[commandchars=\\\{\}]
{\color{incolor}In [{\color{incolor}16}]:} \PY{o}{\PYZpc{}}\PY{k}{matplotlib} notebook
         \PY{n}{fig}\PY{p}{,}\PY{n}{ax} \PY{o}{=} \PY{n}{plt}\PY{o}{.}\PY{n}{subplots}\PY{p}{(}\PY{p}{)}
         \PY{n}{ax}\PY{o}{.}\PY{n}{plot}\PY{p}{(}\PY{n}{wave}\PY{p}{,}\PY{n}{flux}\PY{p}{)}
         \PY{n}{ax}\PY{o}{.}\PY{n}{set\PYZus{}ylim}\PY{p}{(}\PY{l+m+mi}{0}\PY{p}{,}\PY{l+m+mf}{1.1}\PY{p}{)}
         \PY{n}{ax}\PY{o}{.}\PY{n}{set\PYZus{}title}\PY{p}{(}\PY{l+s+s1}{\PYZsq{}}\PY{l+s+s1}{HD122563}\PY{l+s+s1}{\PYZsq{}}\PY{p}{)}
         
         \PY{c+c1}{\PYZsh{} You can use the following command to automatically zoom in on a wavelength range }
         \PY{c+c1}{\PYZsh{} or use the zoom to rectangle button below}
         
         \PY{n}{ax}\PY{o}{.}\PY{n}{set\PYZus{}xlim}\PY{p}{(}\PY{l+m+mi}{4330}\PY{p}{,}\PY{l+m+mi}{4350}\PY{p}{)}
\end{Verbatim}


    
    \begin{verbatim}
<IPython.core.display.Javascript object>
    \end{verbatim}

    
    
    \begin{verbatim}
<IPython.core.display.HTML object>
    \end{verbatim}

    
\begin{Verbatim}[commandchars=\\\{\}]
{\color{outcolor}Out[{\color{outcolor}16}]:} (4330, 4350)
\end{Verbatim}
            
    H \(\beta\) = 4861.36 Angstrom

H \(\gamma\) = 4340.46 Angstrom

    

    \begin{Verbatim}[commandchars=\\\{\}]
{\color{incolor}In [{\color{incolor}17}]:} \PY{c+c1}{\PYZsh{}Compute the radial velocity here:}
         \PY{c+c1}{\PYZsh{}remove hashes and enter relevant values to run the code and print the radial velocity}
         \PY{c+c1}{\PYZsh{} Try doing this for all three lines to get an average and standard deviation as an estimate of the error.}
         
         \PY{n}{c} \PY{o}{=} \PY{l+m+mf}{2.99793e5}
         \PY{n}{lamb\PYZus{}obs} \PY{o}{=} \PY{l+m+mf}{4339.96}
         \PY{n}{lamb\PYZus{}rest} \PY{o}{=} \PY{l+m+mf}{4340.46}
         
         \PY{n}{V\PYZus{}rad} \PY{o}{=} \PY{n}{c} \PY{o}{*} \PY{p}{(}\PY{n}{lamb\PYZus{}obs} \PY{o}{\PYZhy{}} \PY{n}{lamb\PYZus{}rest}\PY{p}{)}\PY{o}{/}\PY{p}{(}\PY{n}{lamb\PYZus{}rest}\PY{p}{)}
         
         \PY{n+nb}{print}\PY{p}{(}\PY{n}{V\PYZus{}rad}\PY{p}{)}
\end{Verbatim}


    \begin{Verbatim}[commandchars=\\\{\}]
-34.53470369500007

    \end{Verbatim}

    This equation can be re-arranged to apply a shift to the observed
spectrum to achieve a rest wavelength state. Hint: make sure you use the
correct sign to the radial velocity to produce the correct redshift or
blueshift as appropriate. You can check you've made the correct
adjustment by re-plotting the spectrum in the window above. If you make
a mistake and what the original spectrum, just reload the file using:
wave,flux = open\_file(star\_id1)

    \begin{Verbatim}[commandchars=\\\{\}]
{\color{incolor}In [{\color{incolor}18}]:} \PY{c+c1}{\PYZsh{}Make the radial velocity correction to the wavelength array}
         
         \PY{n}{wave\PYZus{}new} \PY{o}{=} \PY{n}{wave} \PY{o}{+}\PY{o}{\PYZhy{}} \PY{n}{wave}\PY{o}{*}\PY{n}{V\PYZus{}rad}\PY{o}{/}\PY{n}{c}
         
         \PY{o}{\PYZpc{}}\PY{k}{matplotlib} notebook
         \PY{n}{fig}\PY{p}{,}\PY{n}{ax} \PY{o}{=} \PY{n}{plt}\PY{o}{.}\PY{n}{subplots}\PY{p}{(}\PY{p}{)}
         \PY{n}{ax}\PY{o}{.}\PY{n}{plot}\PY{p}{(}\PY{n}{wave\PYZus{}new}\PY{p}{,}\PY{n}{flux}\PY{p}{)}
         \PY{n}{ax}\PY{o}{.}\PY{n}{set\PYZus{}ylim}\PY{p}{(}\PY{l+m+mi}{0}\PY{p}{,}\PY{l+m+mf}{1.1}\PY{p}{)}
         \PY{n}{ax}\PY{o}{.}\PY{n}{set\PYZus{}title}\PY{p}{(}\PY{l+s+s1}{\PYZsq{}}\PY{l+s+s1}{HD122563}\PY{l+s+s1}{\PYZsq{}}\PY{p}{)}
         
         \PY{c+c1}{\PYZsh{} You can use the following command to automatically zoom in on a wavelength range }
         \PY{c+c1}{\PYZsh{} or use the zoom to rectangle button below}
         
         \PY{n}{ax}\PY{o}{.}\PY{n}{set\PYZus{}xlim}\PY{p}{(}\PY{l+m+mi}{4330}\PY{p}{,}\PY{l+m+mi}{4350}\PY{p}{)}
\end{Verbatim}


    
    \begin{verbatim}
<IPython.core.display.Javascript object>
    \end{verbatim}

    
    
    \begin{verbatim}
<IPython.core.display.HTML object>
    \end{verbatim}

    
\begin{Verbatim}[commandchars=\\\{\}]
{\color{outcolor}Out[{\color{outcolor}18}]:} (4330, 4350)
\end{Verbatim}
            
    Question 3: Radial velocities stretch and shift the spectrum along the
wavelength axis. What happens to a spectrum when the source is spinning
rapidly?

     redshifted 

    \section{PART 3 - METALLICITY}\label{part-3---metallicity}

    Metallicity is a measure of how many 'metals' or heavy elements are in a
star compared to Hydrogen. Metal enrichment happens over many
generations of stars and so the Sun is quite metal rich compared to
older stars that are found for example in the Galactic bulge or halo. In
this section we are going to measure the metallicity of HD122563 and
compare it to the Solar value.

    Question 4: Iron content compared to Hydrogen is used as a proxy for
metallicity. Explain why this is a reasonable approximation.

     We compare Iron to Hydrogen to see the relative depletion of hydrogen
in the stars, the more heavry elements (higher fraction of Iron) the
longer the star has been undergoing fusion on the main sequence 

    In this part we are going to measure the equivalent width (EW) of iron
lines to calculate the {[}Fe/H{]} which is an abbrieviated notation for
the Fe/H ratio compared to the Solar value on a log scale. You can
measure EW using plot\_n\_measure\_EW function.

Remember: EW = \(\int{(1 - F_{\lambda}/F_0)d\lambda}\) (where \(F_0\) is
the flux of the continuum).

The function below returns the measured EW for a spectrum, you can call
it on the following line to measure the EW of the Ca II line shown in
the plot. For those interested you can use this as a quantitative
measure to compare HD122563's Na line against the Sun's.

    \begin{Verbatim}[commandchars=\\\{\}]
{\color{incolor}In [{\color{incolor}33}]:} \PY{n}{plot\PYZus{}n\PYZus{}measure\PYZus{}EW\PYZus{}spectrum}\PY{p}{(}\PY{n}{wave}\PY{p}{,}\PY{n}{flux}\PY{p}{,}\PY{p}{[}\PY{l+m+mf}{5889.0}\PY{p}{,}\PY{l+m+mf}{5890.6}\PY{p}{]}\PY{p}{)}
         \PY{n}{midpoint} \PY{o}{=} \PY{p}{(}\PY{l+m+mf}{5889.959}\PY{o}{+}\PY{l+m+mf}{5895.937}\PY{p}{)} \PY{o}{/}\PY{l+m+mi}{2} 
         \PY{n+nb}{print}\PY{p}{(}\PY{l+s+s1}{\PYZsq{}}\PY{l+s+s1}{midpoint wavelength: }\PY{l+s+s1}{\PYZsq{}} \PY{o}{+} \PY{n+nb}{str}\PY{p}{(}\PY{n}{midpoint}\PY{p}{)}\PY{p}{)}
\end{Verbatim}


    
    \begin{verbatim}
<IPython.core.display.Javascript object>
    \end{verbatim}

    
    
    \begin{verbatim}
<IPython.core.display.HTML object>
    \end{verbatim}

    
    \begin{Verbatim}[commandchars=\\\{\}]
Equivalent Width = 0.2 A
midpoint wavelength: 5892.948

    \end{Verbatim}

    To find a list of potential iron lines, let's load a line list file
using the function 'read\_n\_display', which takes the filename as an
argument, reads the file and prints it to the screen. The sample line
list is called: 'example.ew'

The line list file has columns that have various characteristics of that
line feature. For now let's focus on the first two columns, the
wavelength of the line in angstroms and the element of the feature
(atomic number).

    \begin{Verbatim}[commandchars=\\\{\}]
{\color{incolor}In [{\color{incolor}34}]:} \PY{n}{read\PYZus{}n\PYZus{}display}\PY{p}{(}\PY{l+s+s1}{\PYZsq{}}\PY{l+s+s1}{example.ew}\PY{l+s+s1}{\PYZsq{}}\PY{p}{)}
\end{Verbatim}


    \begin{Verbatim}[commandchars=\\\{\}]
  3302.369      11.0      0.00    -1.745               XXX.X
  3302.979      11.0      0.00    -2.050               XXX.X
  5889.959      11.0      0.00     0.108               XXX.X  
  5895.937      11.0      0.00    -0.194               XXX.X
  3829.360      12.0      2.71    -0.227               XXX.X
  4057.505      12.0      4.35    -0.900               XXX.X
  5172.691      12.0      2.71    -0.393               XXX.X
  5183.604      12.0      2.72    -0.167               XXX.X
  5528.416      12.0      4.35    -0.498               XXX.X
  5711.102      12.0      4.35    -1.724               XXX.X
  3943.998      13.0      0.00    -0.638               XXX.X
  3961.525      13.0      0.01    -0.340               XXX.X
  3853.660      14.0      6.85    -1.341               XXX.X
  3905.523      14.0      1.91    -1.041               XXX.X
  4102.936      14.0      1.91    -2.470               XXX.X
  5645.611      14.0      4.93    -1.630               XXX.X
  4289.364      20.0      1.88    -0.300               XXX.X
  4302.536      20.0      1.90     0.276               XXX.X
  4318.650      20.0      1.90    -0.210               XXX.X
  4425.437      20.0      1.88    -0.358               XXX.X
  4435.677      20.0      1.89    -0.519               XXX.X
  4454.774      20.0      1.90     0.260               XXX.X
  4578.551      20.0      2.52    -0.558               XXX.X
  5581.975      20.0      2.52    -0.710               XXX.X
  5588.752      20.0      2.53     0.210               XXX.X
  5590.113      20.0      2.52    -0.710               XXX.X
  5594.464      20.0      2.52    -0.050               XXX.X
  5601.290      20.0      2.53    -0.690               XXX.X
  5857.452      20.0      2.93     0.230               XXX.X
  4314.080      21.1      0.62    -0.100               XXX.X
  4320.742      21.1      0.60    -0.250               XXX.X
  5526.785      21.1      1.77     0.020               XXX.X
  5667.149      21.1      1.50    -1.240               XXX.X
  5669.038      21.1      1.50    -1.120               XXX.X
  5684.195      21.1      1.51    -1.080               XXX.X
  3924.529      22.0      0.02    -0.937               XXX.X
  3958.210      22.0      0.05    -0.176               XXX.X
  4512.739      22.0      0.84    -0.480               XXX.X
  4518.027      22.0      0.83    -0.324               XXX.X
  4533.242      22.0      0.85     0.476               XXX.X
  4534.779      22.0      0.84     0.280               XXX.X
  4548.765      22.0      0.83    -0.354               XXX.X
  4555.488      22.0      0.85    -0.488               XXX.X
  4981.735      22.0      0.85     0.504               XXX.X
  4999.503      22.0      0.83     0.250               XXX.X
  5022.869      22.0      0.83    -0.434               XXX.X
  5039.957      22.0      0.02    -1.130               XXX.X
  5173.742      22.0      0.00    -1.118               XXX.X
  5210.384      22.0      0.05    -0.883               XXX.X
  5866.452      22.0      1.07    -0.840               XXX.X
  3900.540      22.1      1.13    -0.440               XXX.X
  4300.043      22.1      1.18    -0.770               XXX.X
  4312.879      22.1      1.18    -1.160               XXX.X
  4395.032      22.1      1.08    -0.660               XXX.X
  4399.765      22.1      1.23    -1.270               XXX.X
  4417.714      22.1      1.17    -1.430               XXX.X
  4443.806      22.1      1.08    -0.710               XXX.X
  4468.495      22.1      1.13    -0.620               XXX.X
  4501.275      22.1      1.12    -0.750               XXX.X
  4533.969      22.1      1.24    -0.770               XXX.X
  4563.762      22.1      1.22    -0.960               XXX.X
  4571.974      22.1      1.57    -0.520               XXX.X
  4583.412      22.1      1.16    -2.870               XXX.X
  4589.949      22.1      1.23    -1.780               XXX.X
  4762.767      22.1      1.08    -2.710               XXX.X
  4865.613      22.1      1.12    -2.810               XXX.X
  4911.193      22.1      3.12    -0.340               XXX.X
  5185.900      22.1      1.89    -1.460               XXX.X
  3545.198      23.1      1.10    -0.390               XXX.X
  3592.023      23.1      1.10    -0.370               XXX.X
  4274.804      24.0      0.00    -0.230               XXX.X
  4289.723      24.0      0.00    -0.360               XXX.X
  4545.957      24.0      0.94    -1.380               XXX.X
  4600.758      24.0      1.00    -1.260               XXX.X
  4613.365      24.0      0.96    -1.680               XXX.X
  4616.128      24.0      0.98    -1.180               XXX.X
  4626.177      24.0      0.97    -1.320               XXX.X
  4652.162      24.0      1.00    -1.030               XXX.X
  5206.042      24.0      0.94     0.019               XXX.X
  5208.449      24.0      0.94     0.159               XXX.X
  5298.273      24.0      0.98    -1.160               XXX.X
  5345.799      24.0      1.00    -0.980               XXX.X
  5348.322      24.0      1.00    -1.290               XXX.X
  5409.789      24.0      1.03    -0.720               XXX.X
  4754.040      25.0      2.28    -0.085               XXX.X
  4783.425      25.0      2.30     0.042               XXX.X
  5394.670      25.0      0.00    -3.503               XXX.X
  3441.991      25.1      1.78    -0.273               XXX.X
  3460.330      25.1      1.81    -0.540               XXX.X
  3488.676      25.1      1.85    -0.860               XXX.X
  3749.485	26.0	  0.91     0.161               XXX.X
  3787.879	26.0	  1.01    -0.859               XXX.X
  3841.047      26.0      1.60    -0.045               XXX.X
  3865.528      26.0      1.01    -0.982               XXX.X
  3886.287      26.0      0.05    -1.076               XXX.X
  3887.049      26.0      0.92    -1.144               XXX.X
  3891.923      26.0      3.42    -0.734               XXX.X
  3895.656      26.0      0.11    -1.670               XXX.X
  3899.711      26.0      0.09    -1.531               XXX.X
  3902.948      26.0      1.56    -0.466               XXX.X
  3906.484      26.0      0.11    -2.243               XXX.X
  3916.730      26.0      3.23    -0.604               XXX.X
  3920.263      26.0      0.12    -1.746               XXX.X
  3922.916      26.0      0.05    -1.651               XXX.X
  3930.305      26.0      0.09    -1.491               XXX.X
  3949.954      26.0      2.18    -1.251               XXX.X
  4271.158      26.0      2.45    -0.349               XXX.X
  4271.768      26.0      1.49    -0.164               XXX.X
  4282.404      26.0      2.18    -0.779               XXX.X
  4294.114      26.0      1.49    -1.110               XXX.X
  4299.223      26.0      2.42    -0.405               XXX.X
  4307.902      26.0      1.56    -0.072               XXX.X
  4325.765      26.0      1.61     0.006               XXX.X
  4337.046      26.0      1.56    -1.695               XXX.X
  4375.934      26.0      0.00    -3.031               XXX.X
  4383.547      26.0      1.49     0.200               XXX.X
  4404.752      26.0      1.56    -0.142               XXX.X
  4415.123      26.0      1.61    -0.615               XXX.X
  4427.317      26.0      0.05    -3.044               XXX.X
  4430.616      26.0      2.22    -1.659               XXX.X
  4442.340      26.0      2.20    -1.255               XXX.X
  4447.718      26.0      2.22    -1.342               XXX.X
  4459.107      26.0      2.18    -1.279               XXX.X
  4461.656      26.0      0.09    -3.210               XXX.X
  4489.744      26.0      0.12    -3.966               XXX.X
  4494.565      26.0      2.20    -1.136               XXX.X
  4531.149      26.0      1.49    -2.155               XXX.X
  4592.651      26.0      1.56    -2.449               XXX.X
  4602.946      26.0      1.49    -2.220               XXX.X
  4625.048      26.0      3.24    -1.348               XXX.X
  4871.317      26.0      2.87    -0.362               XXX.X
  4872.137      26.0      2.88    -0.567               XXX.X
  4891.491      26.0      2.85    -0.112               XXX.X
  4918.998      26.0      2.87    -0.342               XXX.X
  4920.509      26.0      2.83     0.068               XXX.X
  4957.599      26.0      2.81     0.233               XXX.X
  5166.286      26.0      0.00    -4.195               XXX.X
  5171.600      26.0      1.49    -1.793               XXX.X
  5192.343      26.0      3.00    -0.421               XXX.X
  5194.944      26.0      1.56    -2.090               XXX.X
  5198.713      26.0      2.22    -2.135               XXX.X
  5217.387      26.0      3.21    -1.162               XXX.X
  5227.188      26.0      1.56    -1.228               XXX.X
  5232.941      26.0      2.94    -0.057               XXX.X
  5269.539      26.0      0.86    -1.321               XXX.X
  5324.181      26.0      3.21    -0.103               XXX.X
  5393.171      26.0      3.24    -0.715               XXX.X
  5405.782      26.0      0.99    -1.844               XXX.X
  5410.910      26.0      4.47     0.398               XXX.X
  5415.200      26.0      4.39     0.642               XXX.X
  5424.066      26.0      4.32     0.510               XXX.X
  5434.527      26.0      1.01    -2.122               XXX.X
  5445.042      26.0      4.39    -0.030               XXX.X
  5473.901      26.0      4.15    -0.790               XXX.X
  5497.518      26.0      1.01    -2.849               XXX.X
  5501.466      26.0      0.96    -3.047               XXX.X
  5506.781      26.0      0.99    -2.797               XXX.X
  5569.623      26.0      3.42    -0.486               XXX.X
  5572.842      26.0      3.40    -0.275               XXX.X
  5576.087      26.0      3.43    -0.940               XXX.X
  5586.756      26.0      3.37    -0.144               XXX.X
  5624.542      26.0      3.42    -0.755               XXX.X
  5662.512      26.0      4.18    -0.573               XXX.X
  5701.556      26.0      2.56    -2.216               XXX.X
  5762.975      26.0      4.21    -0.470               XXX.X
  5956.705      26.0      0.86    -4.608               XXX.X
  3255.901      26.1      0.99    -2.498               XXX.X
  3277.348      26.1      0.99    -2.191               XXX.X
  3281.291      26.1      1.04    -2.678               XXX.X
  4416.817      26.1      2.77    -2.430               XXX.X
  4491.401      26.1      2.84    -2.600               XXX.X
  4508.300      26.1      2.84    -2.280               XXX.X
  4555.888      26.1      2.82    -2.170               XXX.X
  4576.333      26.1      2.83    -2.900               XXX.X
  4583.831      26.1      2.81    -1.740               XXX.X
  4923.929      26.1      2.89    -1.206               XXX.X
  5018.442      26.1      2.89    -1.350               XXX.X
  5197.571      26.1      3.23    -2.230               XXX.X
  5234.621      26.1      3.22    -2.220               XXX.X
  5534.847      26.1      3.25    -2.640               XXX.X
  3842.045      27.0      0.92    -0.763               XXX.X
  3845.466      27.0      0.92     0.009               XXX.X
  3873.114      27.0      0.43    -0.666               XXX.X
  3807.145      28.0      0.42    -1.180               XXX.X
  3858.299      28.0      0.42    -0.967               XXX.X
  4401.550      28.0      3.19     0.084               XXX.X
  5578.719      28.0      1.68    -2.640               XXX.X
  5587.849      28.0      1.94    -2.140               XXX.X
  5592.280      28.0      1.95    -2.590               XXX.X
  5892.883      28.0      1.99    -2.340               XXX.X
  3247.540      29.0      0.00    -0.056               XXX.X
  3273.959      29.0      0.00    -0.360               XXX.X
  4607.331      38.0      0.00     0.283               XXX.X
  3774.334      39.1      0.13     0.220               XXX.X
  3788.697      39.1      0.10    -0.060               XXX.X
  3818.350      39.1      0.13    -0.974               XXX.X
  3832.887      39.1      0.18    -0.330               XXX.X
  3950.351      39.1      0.10    -0.484               XXX.X
  4398.013      39.1      0.13    -1.000               XXX.X
  4883.688      39.1      1.08     0.070               XXX.X
  5123.210      39.1      0.99    -0.830               XXX.X
  5200.413      39.1      0.99    -0.570               XXX.X
  5205.734      39.1      1.03    -0.340               XXX.X
  3836.761      40.1      0.56    -0.060               XXX.X
  4317.315      40.1      0.71    -1.380               XXX.X
  4554.033      56.1      0.00     0.140               XXX.X
  4934.076      56.1      0.00    -0.160               XXX.X
  5853.685      56.1      0.60    -0.908               XXX.X
  3964.260      59.1      0.22    -0.400               XXX.X
  4061.090      60.1      0.47     0.290               XXX.X
  3538.522      66.1      0.00    -0.020               XXX.X
  3810.730      67.1      0.00     0.190               XXX.X


    \end{Verbatim}

    \begin{Verbatim}[commandchars=\\\{\}]
{\color{incolor}In [{\color{incolor}35}]:} \PY{c+c1}{\PYZsh{}\PYZsh{}\PYZsh{} Enter your code here \PYZsh{}\PYZsh{} }
         
         \PY{c+c1}{\PYZsh{}\PYZsh{} HINT : You will need to play around with moving the star and stop of each }
         \PY{c+c1}{\PYZsh{} absorbtion line to get the fit only over the absorbtion line }
         
         \PY{n}{plot\PYZus{}n\PYZus{}measure\PYZus{}EW\PYZus{}spectrum}\PY{p}{(}\PY{n}{wave}\PY{p}{,}\PY{n}{flux}\PY{p}{,}\PY{p}{[} \PY{l+m+mf}{3886.6} \PY{p}{,}\PY{l+m+mf}{3887.049}\PY{p}{]}\PY{p}{)}
         \PY{n}{midpoint} \PY{o}{=} \PY{p}{(} \PY{l+m+mf}{3886.287} \PY{o}{+} \PY{l+m+mf}{4583.831}\PY{p}{)} \PY{o}{/} \PY{l+m+mf}{2.0}
         \PY{n+nb}{print}\PY{p}{(}\PY{l+s+s1}{\PYZsq{}}\PY{l+s+s1}{midpoint wavelength: }\PY{l+s+s1}{\PYZsq{}} \PY{o}{+} \PY{n+nb}{str}\PY{p}{(}\PY{n}{midpoint}\PY{p}{)}\PY{p}{)}
         
         
         \PY{c+c1}{\PYZsh{} hint: use plot\PYZus{}n\PYZus{}measure\PYZus{}EW\PYZus{}spectrum(wave,flux,[,]) function to get your EW and }
         \PY{c+c1}{\PYZsh{} midpoint of lambda that you will need below }
\end{Verbatim}


    
    \begin{verbatim}
<IPython.core.display.Javascript object>
    \end{verbatim}

    
    
    \begin{verbatim}
<IPython.core.display.HTML object>
    \end{verbatim}

    
    \begin{Verbatim}[commandchars=\\\{\}]
Equivalent Width = 0.1 A
midpoint wavelength: 4235.059

    \end{Verbatim}

    \begin{Verbatim}[commandchars=\\\{\}]
{\color{incolor}In [{\color{incolor}37}]:} \PY{n}{EW\PYZus{}1\PYZus{}HD122563} \PY{o}{=} \PY{l+m+mf}{0.2} 
         \PY{n}{lambda\PYZus{}HD122563\PYZus{}1} \PY{o}{=} \PY{l+m+mf}{4235.059}
         
         \PY{n}{EW\PYZus{}2\PYZus{}HD122563} \PY{o}{=} \PY{l+m+mf}{0.1} 
         \PY{n}{lambda\PYZus{}HD122563\PYZus{}2} \PY{o}{=} \PY{l+m+mf}{5574.4645}
         
         \PY{n}{EW\PYZus{}3\PYZus{}HD122563} \PY{o}{=} \PY{l+m+mf}{0.1}
         \PY{n}{lambda\PYZus{}HD122563\PYZus{}3} \PY{o}{=}  \PY{l+m+mf}{4919.753}
\end{Verbatim}


    Pick 3 to 5 Fe lines from the line list (lines where the 2nd column has
the number 26) and measure their EWs using HD122563 spectrum below.

    \subsection{Part 3 Finding abundances}\label{part-3-finding-abundances}

We can now use these EW to calculate a value called Epsilon which is
descriebed below:

\begin{equation*}
\epsilon = 12 + log\left(\frac{N_{Fe}}{N_H}\right)
\end{equation*}

This value Epsilon can be used to detemined the stars elemental
abundance. We first need to find the column density of atoms for Iron
(Fe). The column density will be the value of N which we will step
through how to find below

** Before starting the following please have a read through
http://spiff.rit.edu/classes/phys440/lectures/curve/curve.html **

    We will now take the measured EW from above and detemine how many atoms
are present in that line (the value of N) using the below figure, The Y
axis is log(EW /lambda), (EW written above as W) \textbf{This is your
measured EW} And the X axis is log Nf(lambda/5000 Angstroms) where N is
the column density.

\textbf{Note} the line can be descibed by three eqautions as shown
overlaid the plot. The value of N is the eqaution of the X axis

    

    \subsection{Step 1. Find the value of
Y:}\label{step-1.-find-the-value-of-y}

\begin{equation*}
Y = log10(\frac{EW}{\lambda})
\end{equation*}

    \begin{Verbatim}[commandchars=\\\{\}]
{\color{incolor}In [{\color{incolor}38}]:} \PY{c+c1}{\PYZsh{}\PYZsh{} Example on Na}
         \PY{c+c1}{\PYZsh{}\PYZsh{} USE THE BELOW EQUATION \PYZsh{}\PYZsh{} }
         \PY{c+c1}{\PYZsh{}  y = np.log10(EW/central wavelength) }
         
         \PY{n}{Y} \PY{o}{=} \PY{n}{np}\PY{o}{.}\PY{n}{log10}\PY{p}{(}\PY{l+m+mf}{0.2}\PY{o}{/}\PY{l+m+mf}{5892.948}\PY{p}{)}
         \PY{n+nb}{print}\PY{p}{(}\PY{l+s+s1}{\PYZsq{}}\PY{l+s+s1}{Y value for Na: }\PY{l+s+s1}{\PYZsq{}} \PY{o}{+} \PY{n+nb}{str}\PY{p}{(}\PY{n}{Y}\PY{p}{)}\PY{p}{)}
         
         \PY{c+c1}{\PYZsh{}\PYZsh{} Do the same for your 3 to 5 Fe lines below below, where you place }
         \PY{c+c1}{\PYZsh{}\PYZsh{} your EW and lambda for each line: }
         
         \PY{n}{y\PYZus{}Fe\PYZus{}1\PYZus{}HD122563} \PY{o}{=} \PY{n}{np}\PY{o}{.}\PY{n}{log10}\PY{p}{(}\PY{n}{EW\PYZus{}1\PYZus{}HD122563} \PY{o}{/}\PY{n}{lambda\PYZus{}HD122563\PYZus{}1} \PY{p}{)}
         \PY{n+nb}{print}\PY{p}{(}\PY{l+s+s1}{\PYZsq{}}\PY{l+s+s1}{Y value for Fe line 1: }\PY{l+s+s1}{\PYZsq{}} \PY{o}{+} \PY{n+nb}{str}\PY{p}{(}\PY{n}{y\PYZus{}Fe\PYZus{}1\PYZus{}HD122563}\PY{p}{)}\PY{p}{)}
         
         \PY{n}{y\PYZus{}Fe\PYZus{}2\PYZus{}HD122563} \PY{o}{=} \PY{n}{np}\PY{o}{.}\PY{n}{log10}\PY{p}{(}\PY{n}{EW\PYZus{}2\PYZus{}HD122563} \PY{o}{/}\PY{n}{lambda\PYZus{}HD122563\PYZus{}2} \PY{p}{)}
         \PY{n+nb}{print}\PY{p}{(}\PY{l+s+s1}{\PYZsq{}}\PY{l+s+s1}{Y value for Fe line 2: }\PY{l+s+s1}{\PYZsq{}} \PY{o}{+} \PY{n+nb}{str}\PY{p}{(}\PY{n}{y\PYZus{}Fe\PYZus{}2\PYZus{}HD122563}\PY{p}{)}\PY{p}{)}
         
         \PY{n}{y\PYZus{}Fe\PYZus{}2\PYZus{}HD122563} \PY{o}{=} \PY{n}{np}\PY{o}{.}\PY{n}{log10}\PY{p}{(}\PY{n}{EW\PYZus{}3\PYZus{}HD122563} \PY{o}{/}\PY{n}{lambda\PYZus{}HD122563\PYZus{}3} \PY{p}{)}
         \PY{n+nb}{print}\PY{p}{(}\PY{l+s+s1}{\PYZsq{}}\PY{l+s+s1}{Y value for Fe line 2: }\PY{l+s+s1}{\PYZsq{}} \PY{o}{+} \PY{n+nb}{str}\PY{p}{(}\PY{n}{y\PYZus{}Fe\PYZus{}2\PYZus{}HD122563}\PY{p}{)}\PY{p}{)}
\end{Verbatim}


    \begin{Verbatim}[commandchars=\\\{\}]
Y value for Na: -4.469302613187371
Y value for Fe line 1: -4.325829469344172
Y value for Fe line 2: -4.746203154076228
Y value for Fe line 2: -4.69194329922472

    \end{Verbatim}

    \subsection{Step 2 Find the value of
X:}\label{step-2-find-the-value-of-x}

Go back to the plot above, and find what the approximate value of x is
along the line of growth. See example for Ca II below:

** We find the approximate X value is 15.35. **

    \#\# TO DO: Find X value for Fe

    \begin{Verbatim}[commandchars=\\\{\}]
{\color{incolor}In [{\color{incolor}39}]:} \PY{c+c1}{\PYZsh{}\PYZsh{} Save your values for X\PYZus{}Fe below as variables to use later }
         
         \PY{n}{X\PYZus{}Fe\PYZus{}1\PYZus{}HD122563} \PY{o}{=} \PY{l+m+mf}{13.9}
         \PY{n+nb}{print}\PY{p}{(}\PY{l+s+s1}{\PYZsq{}}\PY{l+s+s1}{X value for Fe line 1: }\PY{l+s+s1}{\PYZsq{}} \PY{o}{+} \PY{n+nb}{str}\PY{p}{(}\PY{n}{X\PYZus{}Fe\PYZus{}1\PYZus{}HD122563}\PY{p}{)}\PY{p}{)}
         
         \PY{n}{X\PYZus{}Fe\PYZus{}2\PYZus{}HD122563} \PY{o}{=} \PY{l+m+mf}{12.05}
         \PY{n+nb}{print}\PY{p}{(}\PY{l+s+s1}{\PYZsq{}}\PY{l+s+s1}{X value for Fe line 2: }\PY{l+s+s1}{\PYZsq{}} \PY{o}{+} \PY{n+nb}{str}\PY{p}{(}\PY{n}{X\PYZus{}Fe\PYZus{}2\PYZus{}HD122563}\PY{p}{)}\PY{p}{)}
         
         \PY{n}{X\PYZus{}Fe\PYZus{}3\PYZus{}HD122563} \PY{o}{=} \PY{l+m+mf}{12.25}
         \PY{n+nb}{print}\PY{p}{(}\PY{l+s+s1}{\PYZsq{}}\PY{l+s+s1}{X value for Fe line 3: }\PY{l+s+s1}{\PYZsq{}} \PY{o}{+} \PY{n+nb}{str}\PY{p}{(}\PY{n}{X\PYZus{}Fe\PYZus{}3\PYZus{}HD122563}\PY{p}{)}\PY{p}{)}
\end{Verbatim}


    \begin{Verbatim}[commandchars=\\\{\}]
X value for Fe line 1: 13.9
X value for Fe line 2: 12.05
X value for Fe line 3: 12.25

    \end{Verbatim}

    \subsection{Step 3 Find the value of
N:}\label{step-3-find-the-value-of-n}

This value of N is in atoms per cm\^{}2 (This value for f will not be
completely accuate for all lines of Fe but will do for the propose of
this assignment, we are using the oscillator strength specficlly of the
Fe II transistion at 321.045nm. Oscillator numbers for ions can be found
here http://vizier.u-strasbg.fr/viz-bin/VizieR-3?-source=VI/69/catalog,
where the value of \(gf\) is log(f))

    \#\# TO DO: Find N value for Fe

    \begin{Verbatim}[commandchars=\\\{\}]
{\color{incolor}In [{\color{incolor}44}]:} \PY{c+c1}{\PYZsh{} Fill in the values for X and Lambda below to find your value of N for HD122563,  use the example}
         \PY{c+c1}{\PYZsh{} code of Na as a template }
         
         \PY{c+c1}{\PYZsh{} N = (10**(X))/((f)*((lambda)/(5000)))}
         \PY{c+c1}{\PYZsh{} N = (10**(15.35))/((1.19)*((3933.8)/(5000)))}
         \PY{n}{f} \PY{o}{=} \PY{l+m+mf}{0.020}
         \PY{n}{N\PYZus{}Fe\PYZus{}HD122563\PYZus{}1} \PY{o}{=} \PY{p}{(}\PY{l+m+mi}{10}\PY{o}{*}\PY{o}{*}\PY{p}{(}\PY{n}{X\PYZus{}Fe\PYZus{}1\PYZus{}HD122563}\PY{p}{)}\PY{p}{)}\PY{o}{/}\PY{p}{(}\PY{p}{(}\PY{n}{f}\PY{p}{)}\PY{o}{*}\PY{p}{(}\PY{p}{(}\PY{n}{lambda\PYZus{}HD122563\PYZus{}1}\PY{p}{)}\PY{o}{/}\PY{p}{(}\PY{l+m+mi}{5000}\PY{p}{)}\PY{p}{)}\PY{p}{)}
         \PY{n+nb}{print}\PY{p}{(}\PY{l+s+s1}{\PYZsq{}}\PY{l+s+s1}{N value for Fe line 1: }\PY{l+s+s1}{\PYZsq{}} \PY{o}{+} \PY{n+nb}{str}\PY{p}{(}\PY{n}{N\PYZus{}Fe\PYZus{}HD122563\PYZus{}1}\PY{p}{)}\PY{p}{)}
         
         \PY{n}{N\PYZus{}Fe\PYZus{}HD122563\PYZus{}2} \PY{o}{=} \PY{p}{(}\PY{l+m+mi}{10}\PY{o}{*}\PY{o}{*}\PY{p}{(}\PY{n}{X\PYZus{}Fe\PYZus{}2\PYZus{}HD122563}\PY{p}{)}\PY{p}{)}\PY{o}{/}\PY{p}{(}\PY{p}{(}\PY{n}{f}\PY{p}{)}\PY{o}{*}\PY{p}{(}\PY{p}{(}\PY{n}{lambda\PYZus{}HD122563\PYZus{}2}\PY{p}{)}\PY{o}{/}\PY{p}{(}\PY{l+m+mi}{5000}\PY{p}{)}\PY{p}{)}\PY{p}{)}
         \PY{n+nb}{print}\PY{p}{(}\PY{l+s+s1}{\PYZsq{}}\PY{l+s+s1}{N value for Fe line 2: }\PY{l+s+s1}{\PYZsq{}} \PY{o}{+} \PY{n+nb}{str}\PY{p}{(}\PY{n}{N\PYZus{}Fe\PYZus{}HD122563\PYZus{}2}\PY{p}{)}\PY{p}{)}
         
         \PY{n}{N\PYZus{}Fe\PYZus{}HD122563\PYZus{}3} \PY{o}{=} \PY{p}{(}\PY{l+m+mi}{10}\PY{o}{*}\PY{o}{*}\PY{p}{(}\PY{n}{X\PYZus{}Fe\PYZus{}2\PYZus{}HD122563}\PY{p}{)}\PY{p}{)}\PY{o}{/}\PY{p}{(}\PY{p}{(}\PY{n}{f}\PY{p}{)}\PY{o}{*}\PY{p}{(}\PY{p}{(}\PY{n}{lambda\PYZus{}HD122563\PYZus{}2}\PY{p}{)}\PY{o}{/}\PY{p}{(}\PY{l+m+mi}{5000}\PY{p}{)}\PY{p}{)}\PY{p}{)}
         \PY{n+nb}{print}\PY{p}{(}\PY{l+s+s1}{\PYZsq{}}\PY{l+s+s1}{N value for Fe line 3: }\PY{l+s+s1}{\PYZsq{}} \PY{o}{+} \PY{n+nb}{str}\PY{p}{(}\PY{n}{N\PYZus{}Fe\PYZus{}HD122563\PYZus{}3}\PY{p}{)}\PY{p}{)}
\end{Verbatim}


    \begin{Verbatim}[commandchars=\\\{\}]
N value for Fe line 1: 4689003357003304.0
N value for Fe line 2: 50319562278222.66
N value for Fe line 3: 50319562278222.66

    \end{Verbatim}

    \section{Now redo the same steps and find the value for the solar twin
HD147513}\label{now-redo-the-same-steps-and-find-the-value-for-the-solar-twin-hd147513}

We will start you off ...

    \begin{Verbatim}[commandchars=\\\{\}]
{\color{incolor}In [{\color{incolor}58}]:} \PY{c+c1}{\PYZsh{}\PYZsh{} do the same steps }
         
         \PY{c+c1}{\PYZsh{} Pick 3 to 5 Fe lines from the line list for the solar twin }
         \PY{c+c1}{\PYZsh{} hint: plot\PYZus{}n\PYZus{}measure\PYZus{}EW\PYZus{}spectrum(sol\PYZus{}wave,sol\PYZus{}flux,[,]) }
         
         \PY{n}{EW\PYZus{}1\PYZus{}HD147513} \PY{o}{=} \PY{n}{plot\PYZus{}n\PYZus{}measure\PYZus{}EW\PYZus{}spectrum}\PY{p}{(}\PY{n}{sol\PYZus{}wave}\PY{p}{,}\PY{n}{sol\PYZus{}flux}\PY{p}{,}\PY{p}{[} \PY{l+m+mf}{3886.6} \PY{p}{,}\PY{l+m+mf}{3886.9}\PY{p}{]}\PY{p}{)}
         \PY{n}{lambda\PYZus{}HD147513\PYZus{}1} \PY{o}{=} \PY{p}{(} \PY{l+m+mf}{3886.6} \PY{o}{+} \PY{l+m+mf}{3886.9}\PY{p}{)} \PY{o}{/} \PY{l+m+mf}{2.0}
         
         
         \PY{n}{EW\PYZus{}2\PYZus{}HD147513} \PY{o}{=}\PY{n}{plot\PYZus{}n\PYZus{}measure\PYZus{}EW\PYZus{}spectrum}\PY{p}{(}\PY{n}{sol\PYZus{}wave}\PY{p}{,}\PY{n}{sol\PYZus{}flux}\PY{p}{,}\PY{p}{[} \PY{l+m+mf}{4919.3} \PY{p}{,}\PY{l+m+mf}{4919.85}\PY{p}{]}\PY{p}{)}
         \PY{n}{lambda\PYZus{}HD147513\PYZus{}2} \PY{o}{=}  \PY{p}{(} \PY{l+m+mf}{4919.3} \PY{o}{+} \PY{l+m+mf}{4919.85}\PY{p}{)} \PY{o}{/} \PY{l+m+mf}{2.0}
         
         
         \PY{n}{EW\PYZus{}3\PYZus{}HD147513} \PY{o}{=} \PY{n}{plot\PYZus{}n\PYZus{}measure\PYZus{}EW\PYZus{}spectrum}\PY{p}{(}\PY{n}{sol\PYZus{}wave}\PY{p}{,}\PY{n}{sol\PYZus{}flux}\PY{p}{,}\PY{p}{[}\PY{l+m+mf}{5703.1} \PY{p}{,} \PY{l+m+mf}{5703.8} \PY{p}{]}\PY{p}{)}
         \PY{n}{lambda\PYZus{}HD147513\PYZus{}3} \PY{o}{=}  \PY{p}{(}\PY{l+m+mf}{5703.1} \PY{o}{+} \PY{l+m+mf}{5171.82}\PY{p}{)} \PY{o}{/} \PY{l+m+mf}{2.0}
\end{Verbatim}


    
    \begin{verbatim}
<IPython.core.display.Javascript object>
    \end{verbatim}

    
    
    \begin{verbatim}
<IPython.core.display.HTML object>
    \end{verbatim}

    
    \begin{Verbatim}[commandchars=\\\{\}]
Equivalent Width = 0.1 A

    \end{Verbatim}

    
    \begin{verbatim}
<IPython.core.display.Javascript object>
    \end{verbatim}

    
    
    \begin{verbatim}
<IPython.core.display.HTML object>
    \end{verbatim}

    
    \begin{Verbatim}[commandchars=\\\{\}]
Equivalent Width = 0.1 A

    \end{Verbatim}

    
    \begin{verbatim}
<IPython.core.display.Javascript object>
    \end{verbatim}

    
    
    \begin{verbatim}
<IPython.core.display.HTML object>
    \end{verbatim}

    
    \begin{Verbatim}[commandchars=\\\{\}]
Equivalent Width = 0.1 A
EW\_1\_HD147513: None

    \end{Verbatim}

    \begin{Verbatim}[commandchars=\\\{\}]
{\color{incolor}In [{\color{incolor}59}]:} \PY{c+c1}{\PYZsh{}\PYZsh{} Write down the values outputed for EW }
         \PY{n}{EW\PYZus{}1\PYZus{}HD147513} \PY{o}{=} \PY{l+m+mf}{0.1}
         \PY{n}{EW\PYZus{}2\PYZus{}HD147513} \PY{o}{=} \PY{l+m+mf}{0.1}
         \PY{n}{EW\PYZus{}3\PYZus{}HD147513} \PY{o}{=} \PY{l+m+mf}{0.1}
\end{Verbatim}


    \begin{Verbatim}[commandchars=\\\{\}]
{\color{incolor}In [{\color{incolor}60}]:} \PY{c+c1}{\PYZsh{}  FIRST find Y value for your lines }
         \PY{n}{y\PYZus{}Fe\PYZus{}1\PYZus{}HD147513} \PY{o}{=}  \PY{n}{np}\PY{o}{.}\PY{n}{log10}\PY{p}{(}\PY{n}{EW\PYZus{}1\PYZus{}HD147513} \PY{o}{/} \PY{n}{lambda\PYZus{}HD147513\PYZus{}1}\PY{p}{)}
         \PY{n+nb}{print}\PY{p}{(}\PY{l+s+s1}{\PYZsq{}}\PY{l+s+s1}{Y value for Fe line 1: }\PY{l+s+s1}{\PYZsq{}} \PY{o}{+} \PY{n+nb}{str}\PY{p}{(}\PY{n}{y\PYZus{}Fe\PYZus{}1\PYZus{}HD147513}\PY{p}{)}\PY{p}{)}
         
         \PY{n}{y\PYZus{}Fe\PYZus{}2\PYZus{}HD147513} \PY{o}{=} \PY{n}{np}\PY{o}{.}\PY{n}{log10}\PY{p}{(}\PY{n}{EW\PYZus{}2\PYZus{}HD147513} \PY{o}{/}\PY{n}{lambda\PYZus{}HD147513\PYZus{}2} \PY{p}{)}
         \PY{n+nb}{print}\PY{p}{(}\PY{l+s+s1}{\PYZsq{}}\PY{l+s+s1}{Y value for Fe line 2: }\PY{l+s+s1}{\PYZsq{}} \PY{o}{+} \PY{n+nb}{str}\PY{p}{(}\PY{n}{y\PYZus{}Fe\PYZus{}2\PYZus{}HD147513}\PY{p}{)}\PY{p}{)}
         
         \PY{n}{y\PYZus{}Fe\PYZus{}2\PYZus{}HD147513} \PY{o}{=} \PY{n}{np}\PY{o}{.}\PY{n}{log10}\PY{p}{(}\PY{n}{EW\PYZus{}3\PYZus{}HD147513} \PY{o}{/}\PY{n}{lambda\PYZus{}HD147513\PYZus{}3} \PY{p}{)}
         \PY{n+nb}{print}\PY{p}{(}\PY{l+s+s1}{\PYZsq{}}\PY{l+s+s1}{Y value for Fe line 2: }\PY{l+s+s1}{\PYZsq{}} \PY{o}{+} \PY{n+nb}{str}\PY{p}{(}\PY{n}{y\PYZus{}Fe\PYZus{}2\PYZus{}HD147513}\PY{p}{)}\PY{p}{)}
\end{Verbatim}


    \begin{Verbatim}[commandchars=\\\{\}]
Y value for Fe line 1: -4.589586607234768
Y value for Fe line 2: -4.691927585871575
Y value for Fe line 2: -4.735396075140961

    \end{Verbatim}

    \begin{Verbatim}[commandchars=\\\{\}]
{\color{incolor}In [{\color{incolor}64}]:} \PY{c+c1}{\PYZsh{} SECOND find your X value for your lines}
         
         \PY{n}{X\PYZus{}Fe\PYZus{}1\PYZus{}HD147513} \PY{o}{=} \PY{l+m+mf}{13.00}
         \PY{n+nb}{print}\PY{p}{(}\PY{l+s+s1}{\PYZsq{}}\PY{l+s+s1}{X value for Fe line 1: }\PY{l+s+s1}{\PYZsq{}} \PY{o}{+} \PY{n+nb}{str}\PY{p}{(}\PY{n}{X\PYZus{}Fe\PYZus{}1\PYZus{}HD147513}\PY{p}{)}\PY{p}{)}
         
         \PY{n}{X\PYZus{}Fe\PYZus{}2\PYZus{}HD147513} \PY{o}{=} \PY{l+m+mf}{12.40}
         \PY{n+nb}{print}\PY{p}{(}\PY{l+s+s1}{\PYZsq{}}\PY{l+s+s1}{X value for Fe line 2: }\PY{l+s+s1}{\PYZsq{}} \PY{o}{+} \PY{n+nb}{str}\PY{p}{(}\PY{n}{X\PYZus{}Fe\PYZus{}2\PYZus{}HD147513}\PY{p}{)}\PY{p}{)}
         
         \PY{n}{X\PYZus{}Fe\PYZus{}3\PYZus{}HD147513} \PY{o}{=} \PY{l+m+mf}{12.25}
         \PY{n+nb}{print}\PY{p}{(}\PY{l+s+s1}{\PYZsq{}}\PY{l+s+s1}{X value for Fe line 3: }\PY{l+s+s1}{\PYZsq{}} \PY{o}{+} \PY{n+nb}{str}\PY{p}{(}\PY{n}{X\PYZus{}Fe\PYZus{}3\PYZus{}HD147513}\PY{p}{)}\PY{p}{)}
\end{Verbatim}


    \begin{Verbatim}[commandchars=\\\{\}]
X value for Fe line 1: 13.0
X value for Fe line 2: 12.4
X value for Fe line 3: 12.25

    \end{Verbatim}

    \begin{Verbatim}[commandchars=\\\{\}]
{\color{incolor}In [{\color{incolor}66}]:} \PY{c+c1}{\PYZsh{} THIRD Solve for N for each of hour lines}
         
         \PY{c+c1}{\PYZsh{} N = (10**(X))/((f)*((lambda)/(5000)))}
         \PY{c+c1}{\PYZsh{} N = (10**(15.35))/((1.19)*((3933.8)/(5000)))}
         
         \PY{n}{N\PYZus{}Fe\PYZus{}HD147513\PYZus{}1} \PY{o}{=} \PY{p}{(}\PY{l+m+mi}{10}\PY{o}{*}\PY{o}{*}\PY{p}{(}\PY{n}{X\PYZus{}Fe\PYZus{}1\PYZus{}HD147513}\PY{p}{)}\PY{p}{)}\PY{o}{/}\PY{p}{(}\PY{p}{(}\PY{n}{f}\PY{p}{)}\PY{o}{*}\PY{p}{(}\PY{p}{(}\PY{n}{lambda\PYZus{}HD147513\PYZus{}1}\PY{p}{)}\PY{o}{/}\PY{p}{(}\PY{l+m+mi}{5000}\PY{p}{)}\PY{p}{)}\PY{p}{)}
         \PY{n+nb}{print}\PY{p}{(}\PY{l+s+s1}{\PYZsq{}}\PY{l+s+s1}{N value for Fe line 1: }\PY{l+s+s1}{\PYZsq{}} \PY{o}{+} \PY{n+nb}{str}\PY{p}{(}\PY{n}{N\PYZus{}Fe\PYZus{}HD147513\PYZus{}1}\PY{p}{)}\PY{p}{)}
         
         \PY{n}{N\PYZus{}Fe\PYZus{}HD147513\PYZus{}2} \PY{o}{=} \PY{p}{(}\PY{l+m+mi}{10}\PY{o}{*}\PY{o}{*}\PY{p}{(}\PY{n}{X\PYZus{}Fe\PYZus{}2\PYZus{}HD147513}\PY{p}{)}\PY{p}{)}\PY{o}{/}\PY{p}{(}\PY{p}{(}\PY{n}{f}\PY{p}{)}\PY{o}{*}\PY{p}{(}\PY{p}{(}\PY{n}{lambda\PYZus{}HD147513\PYZus{}2}\PY{p}{)}\PY{o}{/}\PY{p}{(}\PY{l+m+mi}{5000}\PY{p}{)}\PY{p}{)}\PY{p}{)}
         \PY{n+nb}{print}\PY{p}{(}\PY{l+s+s1}{\PYZsq{}}\PY{l+s+s1}{N value for Fe line 2: }\PY{l+s+s1}{\PYZsq{}} \PY{o}{+} \PY{n+nb}{str}\PY{p}{(}\PY{n}{N\PYZus{}Fe\PYZus{}HD147513\PYZus{}2}\PY{p}{)}\PY{p}{)}
         
         \PY{n}{N\PYZus{}Fe\PYZus{}HD147513\PYZus{}3} \PY{o}{=} \PY{p}{(}\PY{l+m+mi}{10}\PY{o}{*}\PY{o}{*}\PY{p}{(}\PY{n}{X\PYZus{}Fe\PYZus{}3\PYZus{}HD147513}\PY{p}{)}\PY{p}{)}\PY{o}{/}\PY{p}{(}\PY{p}{(}\PY{n}{f}\PY{p}{)}\PY{o}{*}\PY{p}{(}\PY{p}{(}\PY{n}{lambda\PYZus{}HD147513\PYZus{}3}\PY{p}{)}\PY{o}{/}\PY{p}{(}\PY{l+m+mi}{5000}\PY{p}{)}\PY{p}{)}\PY{p}{)}
         \PY{n+nb}{print}\PY{p}{(}\PY{l+s+s1}{\PYZsq{}}\PY{l+s+s1}{N value for Fe line 3: }\PY{l+s+s1}{\PYZsq{}} \PY{o}{+} \PY{n+nb}{str}\PY{p}{(}\PY{n}{N\PYZus{}Fe\PYZus{}HD147513\PYZus{}3}\PY{p}{)}\PY{p}{)}
\end{Verbatim}


    \begin{Verbatim}[commandchars=\\\{\}]
N value for Fe line 1: 643210908857014.2
N value for Fe line 2: 127647532129786.7
N value for Fe line 3: 81760574332451.31

    \end{Verbatim}

    Finally now we will compare out solar twin values of Fe/H to the sun. We
will find the value of epsilon for each star first using:

\begin{equation*}
\epsilon = 12 + log\left(\frac{N_{Fe}}{N_H}\right)
\end{equation*}

We know, N\_H = 2.7 × 10\(^{19}\) so sub in the value of N\_Fe you found
for both the original star and the solar twin to get the Fe/H ratio

    \begin{Verbatim}[commandchars=\\\{\}]
{\color{incolor}In [{\color{incolor}73}]:} \PY{c+c1}{\PYZsh{}\PYZsh{} Find the average of number of Fe atoms found in the three absorbtion lines used }
         
         \PY{n}{Fe\PYZus{}HD122563\PYZus{}av} \PY{o}{=} \PY{p}{(}\PY{n}{N\PYZus{}Fe\PYZus{}HD122563\PYZus{}1} \PY{o}{+} \PY{n}{N\PYZus{}Fe\PYZus{}HD122563\PYZus{}2} \PY{o}{+} \PY{n}{N\PYZus{}Fe\PYZus{}HD122563\PYZus{}3}\PY{p}{)}\PY{o}{/}\PY{l+m+mi}{3}
         \PY{n+nb}{print}\PY{p}{(}\PY{l+s+s1}{\PYZsq{}}\PY{l+s+s1}{Average of HD122563: }\PY{l+s+s1}{\PYZsq{}} \PY{o}{+} \PY{n+nb}{str}\PY{p}{(}\PY{n}{Fe\PYZus{}HD122563\PYZus{}av}\PY{p}{)}\PY{p}{)}
         
         \PY{n}{Fe\PYZus{}HD147513\PYZus{}av} \PY{o}{=} \PY{p}{(}\PY{n}{N\PYZus{}Fe\PYZus{}HD147513\PYZus{}1} \PY{o}{+} \PY{n}{N\PYZus{}Fe\PYZus{}HD147513\PYZus{}2} \PY{o}{+} \PY{n}{N\PYZus{}Fe\PYZus{}HD147513\PYZus{}3}\PY{p}{)}\PY{o}{/}\PY{l+m+mi}{3}
         \PY{n+nb}{print}\PY{p}{(}\PY{l+s+s1}{\PYZsq{}}\PY{l+s+s1}{Average of HD147513: }\PY{l+s+s1}{\PYZsq{}} \PY{o}{+} \PY{n+nb}{str}\PY{p}{(}\PY{n}{Fe\PYZus{}HD147513\PYZus{}av}\PY{p}{)}\PY{p}{)}
\end{Verbatim}


    \begin{Verbatim}[commandchars=\\\{\}]
Average of HD122563: 1596547493853250.0
Average of HD147513: 284206338439750.75

    \end{Verbatim}

    \begin{Verbatim}[commandchars=\\\{\}]
{\color{incolor}In [{\color{incolor}74}]:} \PY{c+c1}{\PYZsh{}\PYZsh{} Now use the average to solve for e }
         
         \PY{n}{e\PYZus{}HD122563} \PY{o}{=} \PY{l+m+mi}{12} \PY{o}{+} \PY{n}{np}\PY{o}{.}\PY{n}{log10}\PY{p}{(}\PY{n}{Fe\PYZus{}HD122563\PYZus{}av}\PY{o}{/}\PY{p}{(}\PY{l+m+mf}{2.7}\PY{o}{*}\PY{l+m+mi}{10}\PY{o}{*}\PY{o}{*}\PY{p}{(}\PY{l+m+mi}{19}\PY{p}{)}\PY{p}{)}\PY{p}{)}
         \PY{n+nb}{print}\PY{p}{(}\PY{l+s+s1}{\PYZsq{}}\PY{l+s+s1}{Epslion for HD122563: }\PY{l+s+s1}{\PYZsq{}} \PY{o}{+} \PY{n+nb}{str}\PY{p}{(}\PY{n}{e\PYZus{}HD122563}\PY{p}{)}\PY{p}{)}
         
         \PY{n}{e\PYZus{}HD147513} \PY{o}{=} \PY{l+m+mi}{12} \PY{o}{+} \PY{n}{np}\PY{o}{.}\PY{n}{log10}\PY{p}{(}\PY{n}{Fe\PYZus{}HD147513\PYZus{}av}\PY{o}{/}\PY{p}{(}\PY{l+m+mf}{2.7}\PY{o}{*}\PY{l+m+mi}{10}\PY{o}{*}\PY{o}{*}\PY{p}{(}\PY{l+m+mi}{19}\PY{p}{)}\PY{p}{)}\PY{p}{)}
         \PY{n+nb}{print}\PY{p}{(}\PY{l+s+s1}{\PYZsq{}}\PY{l+s+s1}{Epslion for HD147513: }\PY{l+s+s1}{\PYZsq{}} \PY{o}{+} \PY{n+nb}{str}\PY{p}{(}\PY{n}{e\PYZus{}HD147513}\PY{p}{)}\PY{p}{)}
\end{Verbatim}


    \begin{Verbatim}[commandchars=\\\{\}]
Epslion for HD122563: 7.771818078235169
Epslion for HD147513: 7.022269995282961

    \end{Verbatim}

    \subsection{Question: Is the value for epslion larger or smaller for the
solar like
star?}\label{question-is-the-value-for-epslion-larger-or-smaller-for-the-solar-like-star}

     Solar like star HD147513 

    Finally we will compare to the Fe/H values by solving the below:

\begin{equation*}
[Fe/H] = log\left(\frac{N_{Fe}}{N_H}\right) - log\left(\frac{N_{Fe}}{N_H}\right)_{\odot}
\end{equation*}

    \begin{Verbatim}[commandchars=\\\{\}]
{\color{incolor}In [{\color{incolor}78}]:} \PY{c+c1}{\PYZsh{}\PYZsh{} Solve for Fe/H below. Use np.log10() when taking the log of values }
         
         \PY{n}{Fe\PYZus{}H} \PY{o}{=} \PY{n}{np}\PY{o}{.}\PY{n}{log10}\PY{p}{(}\PY{n}{Fe\PYZus{}HD122563\PYZus{}av}\PY{o}{/}\PY{p}{(}\PY{l+m+mf}{2.7}\PY{o}{*}\PY{l+m+mi}{10}\PY{o}{*}\PY{o}{*}\PY{p}{(}\PY{l+m+mi}{19}\PY{p}{)}\PY{p}{)}\PY{p}{)} \PY{o}{\PYZhy{}} \PY{n}{np}\PY{o}{.}\PY{n}{log10}\PY{p}{(}\PY{n}{Fe\PYZus{}HD147513\PYZus{}av}\PY{o}{/}\PY{p}{(}\PY{l+m+mf}{2.7}\PY{o}{*}\PY{l+m+mi}{10}\PY{o}{*}\PY{o}{*}\PY{p}{(}\PY{l+m+mi}{19}\PY{p}{)}\PY{p}{)}\PY{p}{)}
         \PY{n+nb}{print}\PY{p}{(}\PY{l+s+s1}{\PYZsq{}}\PY{l+s+s1}{Fe/H  Value: }\PY{l+s+s1}{\PYZsq{}} \PY{o}{+} \PY{n+nb}{str}\PY{p}{(}\PY{n}{Fe\PYZus{}H}\PY{p}{)}\PY{p}{)}
\end{Verbatim}


    \begin{Verbatim}[commandchars=\\\{\}]
Fe/H  Value: 0.7495480829522085

    \end{Verbatim}

     Question 5: How do your values compare to the literature?

     Type your answer to question 5 here

     Question 6: what could be contributing to the source of error when
measuring EWs and thus elemental abundances? Hint: think about the
preparation for spectra before one even measures EW.

     Type your answer to question 6 here


    % Add a bibliography block to the postdoc
    
    
    
    \end{document}
